% $Header: /home/vedranm/bitbucket/beamer/solutions/generic-talks/generic-ornate-15min-45min.en.tex,v 90e850259b8b 2007/01/28 20:48:30 tantau $
\PassOptionsToPackage{dvipsnames}{xcolor}
\documentclass{beamer}

% This file is a solution template for:

% - Giving a talk on some subject.
% - The talk is between 15min and 45min long.
% - Style is ornate.



% Copyright 2004 by Till Tantau <tantau@users.sourceforge.net>.
%
% In principle, this file can be redistributed and/or modified under
% the terms of the GNU Public License, version 2.
%
% However, this file is supposed to be a template to be modified
% for your own needs. For this reason, if you use this file as a
% template and not specifically distribute it as part of a another
% package/program, I grant the extra permission to freely copy and
% modify this file as you see fit and even to delete this copyright
% notice. 


\mode<presentation>
{
  \usetheme{Frankfurt}
\usecolortheme{beetle}
\setbeamercolor{normal text}{bg=black}
\setbeamercolor{normal text}{fg=white}
\setbeamercolor{frametitle}{fg=white, bg=blue}
\setbeamercolor{title}{fg=orange}
\setbeamercolor{author}{fg=orange}
\setbeamercolor{outline}{fg=red, bg=black}
\setbeamercolor{section in toc}{fg=white, bg=black}
\setbeamercolor{block body}{fg=white, bg=teal}
  % or ...

  %\setbeamercovered{transparent}
  % or whatever (possibly just delete it)
}


\usepackage[english]{babel}
% or whatever

\usepackage[latin1]{inputenc}
% or whatever

\usepackage{graphics}
\usepackage{epsfig}
\usepackage{times}
\usepackage[T1]{fontenc}
\usepackage{hyperref}
\usepackage{arydshln}
\usepackage{ulem}
\definecolor{lightblue}{rgb}{1.0,0.5,0.5}
\definecolor{orange}{rgb}{1.0,0.7,0}

% Or whatever. Note that the encoding and the font should match. If T1
% does not look nice, try deleting the line with the fontenc.

\title[McKay - GSPS - October 24, 2014] % (optional, use only with long paper titles)
{LaTeX: Not Just for Papers!}

\author[Author] % (optional, use only with lots of authors)
{Adam McKay\inst{1}}

\date[Short Occasion] % (optional)
{October 24, 2014 - GSPS}

\institute[Universities of Somewhere and Elsewhere] % (optional, but mostly needed)
{
  \inst{1} University of Texas Austin/McDonald Observatory
}

\subject{Talks}
% This is only inserted into the PDF information catalog. Can be left
% out. 

% If you have a file called "university-logo-filename.xxx", where xxx
% is a graphic format that can be processed by latex or pdflatex,
% resp., then you can add a logo as follows:

% \pgfdeclareimage[height=0.5cm]{university-logo}{university-logo-filename}
% \logo{\pgfuseimage{university-logo}}



% Delete this, if you do not want the table of contents to pop up at
% the beginning of each subsection:
%\AtBeginSubsection[]
%{
%  \begin{frame}<beamer>{Outline}
%    \tableofcontents[currentsection,currentsubsection]
%  \end{frame}
%}

%\AtBeginSection[]
%{
%  \begin{frame}<beamer>{Outline}
%    \tableofcontents[currentsection,currentsubsection]
%  \end{frame}
%}


% If you wish to uncover everything in a step-wise fashion, uncomment
% the following command: 

%\beamerdefaultoverlayspecification{<+->}


\begin{document}
{
\usebackgroundtemplate{\includegraphics[width=\paperwidth,height=\paperheight]{../lovejoy}}

\begin{frame}
  \titlepage
\end{frame}
}

%\begin{frame}{Outline}
 %\tableofcontents
  % You might wish to add the option [pausesections]
%\end{frame}


% Since this a solution template for a generic talk, very little can
% be said about how it should be structured. However, the talk length
% of between 15min and 45min and the theme suggest that you stick to
% the following rules:  

% - Exactly two or three sections (other than the summary).
% - At *most* three subsections per section.
% - Talk about 30s to 2min per frame. So there should be between about
%   15 and 30 frames, all told.

\begin{frame}{Outline}
 \tableofcontents
  % You might wish to add the option [pausesections]
\end{frame}

\section{First}
\subsection{Subsection}

\begin{frame}{What I do}
\begin{itemize}
\item High resolution optical spectroscopy
\item Lots of experience on ARCES at ARC 3.5-meter telescope at Apache Point Observatory, learning to use 2DCoude on the McDonald 107-inch
\item Using forbidden oxygen line emission as a proxy for CO$_2$ in comets
\item Understanding the photochemistry of the coma (comparison of optical and IR spectra)
\end{itemize}
\end{frame}

\begin{frame}{A Common Misconception}
 \begin{itemize}
  \item Q: Why are comets boring to study?
\pause
\item A: Because they are always in a coma!
\pause
\item Actually, the fact that they are in a coma makes them very interesting and challenging objects to study!
 \end{itemize}
\end{frame}

\begin{frame}{Case in Point: Comet ISON}
\begin{itemize}
\item \textcolor{green}{Up:} Upon discovery hailed as ``Comet of the Century''
\item \textcolor{red}{Down:} Underperformed as it came closer to perihelion
\item \textcolor{green}{Up:} Major outbursts gave renewed hope
\item \textcolor{green}{Up:} Models predicted it would survive perihelion
\item \textcolor{red}{Down:} SOHO images showed apparent disintegration event
\item \textcolor{green}{Up:} Something emerged post-perihelion
\item \textcolor{red}{Down:} Didn't survive
\end{itemize}
\end{frame}

{
\setbeamercolor{background canvas}{bg=red}
\begin{frame}{CO$_2$/H$_2$O from Atomic Oxygen}
 \begin{itemize}
 \item Photodissociation:\\ 
 H$_2$O + photon $\rightarrow$ H$_2$ + O($^1$D)\\ 
 CO$_2$ + photon $\rightarrow$ CO + O($^1$S)\\
  \item H$_2$O, CO$_2$, and CO release $^1$S and $^1$D OI with different efficiencies
  \begin{equation}
R\equiv\frac{N(O(^1S))}{N(O(^1D))}=\frac{I_{5577}}{I_{6300}+I_{6364}}
\end{equation}
\item H$_2$O dominated case gives R=0.05-0.1, CO$_2$/CO dominated case gives R=0.6-0.8
\end{itemize}
\end{frame}
}

\begin{frame}{CO$_2$/H$_2$O from Atomic Oxygen}
CO$_2$/H$_2$O ratio given by (McKay et al. 2012):
\begin{equation}
\textcolor{yellow}{\frac{N_{CO_2}}{N_{H_2O}}} = \frac{\textcolor{cyan}{R}\textcolor{lightblue}{W^{^1D}_{H_2O}}-\textcolor{lightblue}{W^{^1S}_{H_2O}}-\textcolor{cyan}{\frac{N_{CO}}{N_{H_2O}}}(\textcolor{lightblue}{W^{^1S}_{CO}}-\textcolor{cyan}{R}\textcolor{lightblue}{W^{^1D}_{CO}})}{\textcolor{lightblue}{W^{^1S}_{CO_2}}-\textcolor{cyan}{R}\textcolor{lightblue}{W^{^1D}_{CO_2}}}
\end{equation} 
$N$=column density\\
$R$=oxygen line ratio\\
$W$=release rate (not well known!)\\
\end{frame}

\begin{frame}{Observations}
ARCES: $\frac{\lambda}{\Delta\lambda}$=31,000\\
HIRES: $\frac{\lambda}{\Delta\lambda}$=47,000\\
McDonald 2DCoude: $\frac{\lambda}{\Delta\lambda}$=60,000\\
\begin{block}{}
\small
\begin{tabular}{llllll} 
UT Date & Instrument & Telescope & $R$ (AU) & $\Delta$ (AU) & $\dot{\Delta}$ (km/s)\\
\hline
Oct 3 & ARCES & APO & 1.61 & 2.09 & -50.9\\
Oct 18-21 & 2DCoude & McDonald & 1.29 & 1.61 & -51.9\\
Oct 25 & HIRESb & Keck I & 1.16 & 1.43 & -50.9\\
Oct 28 & HIRESb & Keck I & 1.09 & 1.34 & -50.0\\
Nov 6 & ARCES & APO & 0.87 & 1.10 & -43.5\\
Nov 15 & ARCES & APO & 0.62 & 0.91 & -25.7\\
Nov 20 & ARCES & APO & 0.46 & 0.86 & -6.2\\
\hline
\end{tabular}\\
\end{block}
\end{frame}

\section{Second}
\subsection{Subsection 2}
\begin{frame}{What are Comets?}
\begin{columns}
 \column{0.5\textwidth}
\begin{itemize}
  \item Consists of a small body of ice and dust called the nucleus when far from the Sun
\item Surface ices begin to sublimate as the comet moves towards the Sun
  \end{itemize}
\column{0.5\textwidth}
\begin{figure}
\includegraphics[height=5.5cm]{Tempel}\\
\tiny{Image Credit: NASA JPL}
\end{figure}
\end{columns}
\end{frame}

\begin{frame}{Cat Pictures!}
 \begin{figure}
\includegraphics[width=9.0cm]{catturkey}  
\end{figure}
\end{frame}

\begin{frame}{More Cat Pictures!}
\begin{columns}
\column{0.25\textwidth}
\begin{figure}
\includegraphics[width=2.0cm]{catturkey}  
\end{figure}
\large{This cat is hungry!}
\column{0.25\textwidth}
\begin{figure}
\includegraphics[width=2.0cm]{catturkey}  
\end{figure}
\footnotesize{This cat is hungry!}
\column{0.25\textwidth}
\begin{figure}
\includegraphics[width=2.0cm]{catturkey}  
\end{figure}
\huge{This cat is hungry!}
\column{0.25\textwidth}
\begin{figure}
\includegraphics[width=2.0cm]{catturkey}  
\end{figure}
\tiny{This cat is hungry!}
\end{columns}
\end{frame}

\begin{frame}{More Cat Pictures!}
\begin{columns}
\column{0.25\textwidth}
No brains are safe.
\column{0.75\textwidth}
\begin{figure}
\includegraphics[width=6.0cm]{zombiecat}  
\end{figure}
\end{columns}
\end{frame}

\begin{frame}{Case in Point: Comet ISON}
\begin{columns}
\column{0.5\textwidth}
\begin{itemize}
 \item Discovered in September 2012 outside of Jupiter's orbit
 \item Found to be on a sungrazing orbit, created lots of hype as ``Comet of the Century''
\end{itemize}
\column{0.5\textwidth}
\begin{figure}
 \includegraphics[width=3.5cm]{Comet_ISON_discovery}\\
 \includegraphics[width=4.0cm]{MeiMei_closet}
\end{figure}
\end{columns}
\end{frame}

\begin{frame}[label=cometsarecats]{Of Comets and Cats}
``Comets are like cats: they have tails, and they do precisely what they want.'' - David Levy 
\begin{columns}
 \column{0.5\textwidth}
 \begin{figure}
 \includegraphics[width=5.0cm]{ison}
 \end{figure}
 \column{0.5\textwidth}
 \begin{figure}
  \includegraphics[width=5.0cm]{MeiMei}
 \end{figure}
 \end{columns}
\end{frame}
 
\subsection{Subsection 3}
\begin{frame}{Case in Point: Comet ISON}
\begin{columns}
\column{0.5\textwidth}
\begin{figure}
 \includegraphics[width=5.0cm]{ISON_HST}
\end{figure}
\column{0.5\textwidth}
\begin{figure}
 \includegraphics[width=5.0cm]{flatcat}
\end{figure}
\end{columns}
\begin{itemize}
 \item Light curve flattened, not getting brighter even though it was moving toward the Sun
 \item Lost behind the Sun in Summer 2013, no observations available until August
\end{itemize}
 \end{frame}

\begin{frame}{Case in Point: Comet ISON}
\begin{columns}
\column{0.6\textwidth}
\begin{figure}
 \includegraphics[width=4.0cm]{catlap}\\
 \includegraphics[width=4.0cm]{catlap2}
\end{figure}
\column{0.4\textwidth}
\begin{figure}
 \includegraphics[width=4.0cm]{catlaptop}
\end{figure}
Comets are like cats!\hyperlink{cometsarecats}{\beamergotobutton{}}
\end{columns}
\end{frame}

\section{Fin}
\subsection{}
\begin{frame}
\begin{center}
See you at Crown!
\end{center}
\begin{figure}
 \includegraphics[width=8.0cm]{beer}
\end{figure}
\end{frame}

\end{document}